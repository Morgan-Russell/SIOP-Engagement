% Options for packages loaded elsewhere
\PassOptionsToPackage{unicode}{hyperref}
\PassOptionsToPackage{hyphens}{url}
%
\documentclass[
  english,
  man]{apa6}
\usepackage{lmodern}
\usepackage{amsmath}
\usepackage{ifxetex,ifluatex}
\ifnum 0\ifxetex 1\fi\ifluatex 1\fi=0 % if pdftex
  \usepackage[T1]{fontenc}
  \usepackage[utf8]{inputenc}
  \usepackage{textcomp} % provide euro and other symbols
  \usepackage{amssymb}
\else % if luatex or xetex
  \usepackage{unicode-math}
  \defaultfontfeatures{Scale=MatchLowercase}
  \defaultfontfeatures[\rmfamily]{Ligatures=TeX,Scale=1}
\fi
% Use upquote if available, for straight quotes in verbatim environments
\IfFileExists{upquote.sty}{\usepackage{upquote}}{}
\IfFileExists{microtype.sty}{% use microtype if available
  \usepackage[]{microtype}
  \UseMicrotypeSet[protrusion]{basicmath} % disable protrusion for tt fonts
}{}
\makeatletter
\@ifundefined{KOMAClassName}{% if non-KOMA class
  \IfFileExists{parskip.sty}{%
    \usepackage{parskip}
  }{% else
    \setlength{\parindent}{0pt}
    \setlength{\parskip}{6pt plus 2pt minus 1pt}}
}{% if KOMA class
  \KOMAoptions{parskip=half}}
\makeatother
\usepackage{xcolor}
\IfFileExists{xurl.sty}{\usepackage{xurl}}{} % add URL line breaks if available
\IfFileExists{bookmark.sty}{\usepackage{bookmark}}{\usepackage{hyperref}}
\hypersetup{
  pdftitle={Development of a BiFactor Engagement Measure},
  pdfauthor={Morgan Russell1, Casey Osorio-Duffoo2, Renata Garcia Prieto Palacios Roji1, \& John Kulas1},
  pdflang={en-EN},
  pdfkeywords={keywords},
  hidelinks,
  pdfcreator={LaTeX via pandoc}}
\urlstyle{same} % disable monospaced font for URLs
\usepackage{graphicx}
\makeatletter
\def\maxwidth{\ifdim\Gin@nat@width>\linewidth\linewidth\else\Gin@nat@width\fi}
\def\maxheight{\ifdim\Gin@nat@height>\textheight\textheight\else\Gin@nat@height\fi}
\makeatother
% Scale images if necessary, so that they will not overflow the page
% margins by default, and it is still possible to overwrite the defaults
% using explicit options in \includegraphics[width, height, ...]{}
\setkeys{Gin}{width=\maxwidth,height=\maxheight,keepaspectratio}
% Set default figure placement to htbp
\makeatletter
\def\fps@figure{htbp}
\makeatother
\setlength{\emergencystretch}{3em} % prevent overfull lines
\providecommand{\tightlist}{%
  \setlength{\itemsep}{0pt}\setlength{\parskip}{0pt}}
\setcounter{secnumdepth}{-\maxdimen} % remove section numbering
% Make \paragraph and \subparagraph free-standing
\ifx\paragraph\undefined\else
  \let\oldparagraph\paragraph
  \renewcommand{\paragraph}[1]{\oldparagraph{#1}\mbox{}}
\fi
\ifx\subparagraph\undefined\else
  \let\oldsubparagraph\subparagraph
  \renewcommand{\subparagraph}[1]{\oldsubparagraph{#1}\mbox{}}
\fi
% Manuscript styling
\usepackage{upgreek}
\captionsetup{font=singlespacing,justification=justified}

% Table formatting
\usepackage{longtable}
\usepackage{lscape}
% \usepackage[counterclockwise]{rotating}   % Landscape page setup for large tables
\usepackage{multirow}		% Table styling
\usepackage{tabularx}		% Control Column width
\usepackage[flushleft]{threeparttable}	% Allows for three part tables with a specified notes section
\usepackage{threeparttablex}            % Lets threeparttable work with longtable

% Create new environments so endfloat can handle them
% \newenvironment{ltable}
%   {\begin{landscape}\centering\begin{threeparttable}}
%   {\end{threeparttable}\end{landscape}}
\newenvironment{lltable}{\begin{landscape}\centering\begin{ThreePartTable}}{\end{ThreePartTable}\end{landscape}}

% Enables adjusting longtable caption width to table width
% Solution found at http://golatex.de/longtable-mit-caption-so-breit-wie-die-tabelle-t15767.html
\makeatletter
\newcommand\LastLTentrywidth{1em}
\newlength\longtablewidth
\setlength{\longtablewidth}{1in}
\newcommand{\getlongtablewidth}{\begingroup \ifcsname LT@\roman{LT@tables}\endcsname \global\longtablewidth=0pt \renewcommand{\LT@entry}[2]{\global\advance\longtablewidth by ##2\relax\gdef\LastLTentrywidth{##2}}\@nameuse{LT@\roman{LT@tables}} \fi \endgroup}

% \setlength{\parindent}{0.5in}
% \setlength{\parskip}{0pt plus 0pt minus 0pt}

% \usepackage{etoolbox}
\makeatletter
\patchcmd{\HyOrg@maketitle}
  {\section{\normalfont\normalsize\abstractname}}
  {\section*{\normalfont\normalsize\abstractname}}
  {}{\typeout{Failed to patch abstract.}}
\patchcmd{\HyOrg@maketitle}
  {\section{\protect\normalfont{\@title}}}
  {\section*{\protect\normalfont{\@title}}}
  {}{\typeout{Failed to patch title.}}
\makeatother
\shorttitle{BiFactor Engagement}
\keywords{keywords\newline\indent Word count: X}
\DeclareDelayedFloatFlavor{ThreePartTable}{table}
\DeclareDelayedFloatFlavor{lltable}{table}
\DeclareDelayedFloatFlavor*{longtable}{table}
\makeatletter
\renewcommand{\efloat@iwrite}[1]{\immediate\expandafter\protected@write\csname efloat@post#1\endcsname{}}
\makeatother
\usepackage{lineno}

\linenumbers
\usepackage{csquotes}
\ifxetex
  % Load polyglossia as late as possible: uses bidi with RTL langages (e.g. Hebrew, Arabic)
  \usepackage{polyglossia}
  \setmainlanguage[]{english}
\else
  \usepackage[shorthands=off,main=english]{babel}
\fi
\ifluatex
  \usepackage{selnolig}  % disable illegal ligatures
\fi
\newlength{\cslhangindent}
\setlength{\cslhangindent}{1.5em}
\newlength{\csllabelwidth}
\setlength{\csllabelwidth}{3em}
\newenvironment{CSLReferences}[2] % #1 hanging-ident, #2 entry spacing
 {% don't indent paragraphs
  \setlength{\parindent}{0pt}
  % turn on hanging indent if param 1 is 1
  \ifodd #1 \everypar{\setlength{\hangindent}{\cslhangindent}}\ignorespaces\fi
  % set entry spacing
  \ifnum #2 > 0
  \setlength{\parskip}{#2\baselineskip}
  \fi
 }%
 {}
\usepackage{calc}
\newcommand{\CSLBlock}[1]{#1\hfill\break}
\newcommand{\CSLLeftMargin}[1]{\parbox[t]{\csllabelwidth}{#1}}
\newcommand{\CSLRightInline}[1]{\parbox[t]{\linewidth - \csllabelwidth}{#1}\break}
\newcommand{\CSLIndent}[1]{\hspace{\cslhangindent}#1}

\title{Development of a BiFactor Engagement Measure}
\author{Morgan Russell\textsuperscript{1}, Casey Osorio-Duffoo\textsuperscript{2}, Renata Garcia Prieto Palacios Roji\textsuperscript{1}, \& John Kulas\textsuperscript{1}}
\date{}


\authornote{

Add complete departmental affiliations for each author here. Each new line herein must be indented, like this line.

Enter author note here.

Correspondence concerning this article should be addressed to Morgan Russell, Postal address. E-mail: \href{mailto:my@email.com}{\nolinkurl{my@email.com}}

}

\affiliation{\vspace{0.5cm}\textsuperscript{1} Montclair State University\\\textsuperscript{2} Harver}

\abstract{
One or two sentences providing a \textbf{basic introduction} to the field, comprehensible to a scientist in any discipline.

Two to three sentences of \textbf{more detailed background}, comprehensible to scientists in related disciplines.

One sentence clearly stating the \textbf{general problem} being addressed by this particular study.

One sentence summarizing the main result (with the words ``\textbf{here we show}'' or their equivalent).

Two or three sentences explaining what the \textbf{main result} reveals in direct comparison to what was thought to be the case previously, or how the main result adds to previous knowledge.

One or two sentences to put the results into a more \textbf{general context}.

Two or three sentences to provide a \textbf{broader perspective}, readily comprehensible to a scientist in any discipline.
}



\begin{document}
\maketitle

The roots of employee (aka work; e.g., W. Schaufeli \& Bakker, 2010) engagement research likely started with theoretical expansions of forms of employee participation (see, for example, Ferris \& Hellier, 1984) and job involvement (e.g., Elloy, Everett, \& Flynn, 1991). This exploration extended into broader considerations of attitudes and emotions (Staw, Sutton, \& Pelled, 1994) and were informed by further exploration of the dimensionality of constructs such as organizational commitment (Meyer \& Allen, 1991). The 1990's saw focused development and refinement (for example, a dissertation; Leone (1995) or actual semantic reference; Kahn (1990)). Staw, Sutton, and Pelled (1994) investigated the relationships between \emph{positive emotions} and favorable work outcomes, and although they do not use the word, ``engagement,'' their distinction between felt and expressed emotion likely held influence upon the burgeoning interest in the engagement construct.

Kahn (1990) described engaged employees as being physically involved, cognitively vigilant, and emotionally connected. Although occasionally referred to as residing on the opposing pole to \emph{burnout} (Christina Maslach \& Leiter, 2008), these two constructs are currently most commonly conceptualized as being distinct (Timms, Brough, \& Graham, 2012; \textbf{kim\_burnout\_2009?}; \textbf{goering2017not?}; \textbf{schaufeli2008workaholism?}), although certainly not universally (Cole, Walter, Bedeian, \& O'Boyle, 2012; Taris, Ybema, \& Beek, 2017). (\textbf{goering2017not?}) explore nomological networks, concluding that these two constructs have a moderate (negative) association, but also distinct nomological networks. (\textbf{schaufeli2008workaholism?}) investigated both internal and external association indicators, concluding that engagement and burnout (as well as \emph{workaholism}) should be considered three distinct constructs.

Burnout can be defined as a psychological syndrome characterized by exhaustion (low energy), cynicism (low involvement), and inefficacy (low efficacy), which is experienced in response to chronic job stressors (e.g., Leiter \& Maslach, 2004; C. Maslach \& Leiter, 1997). Alternatively, engagement refers to an individual worker's involvement and satisfaction as well as enthusiasm for work (Harter, Schmidt, \& Hayes, 2002). W. B. Schaufeli and Bakker (2003) further specify a ``positive, fulfilling, work-related state of mind that is characterized by vigor, dedication, and absorption'' (p.~74). Via their conceptualization, vigor is described as high levels of energy and mental resilience while working. Dedication refers to being strongly involved in one's work and experiencing a sense of significance, enthusiasm, inspiration, pride, and challenge. Absorption is characterized by being fully concentrated and happily engrossed in one's work, whereby time passes quickly and one has difficulties with detaching oneself from work (Wilmar B. Schaufeli, Salanova, González-Romá, \& Bakker, 2002). This absorption element has been noted as being influenced in conceptual specification by (\textbf{csikszentmihalyi1990flow?})'s concept of ``flow.''

Regarding measurement, Gallup is widely acknowledged as an early pioneer in the measurement of the construct (see, for example, Coffman \& Harter, 1999). The Utrecht Work Engagement Scale (UWES) is another self-report questionnaire developed by W. B. Schaufeli and Bakker (2003) that directly assesses the vigor, dedication, and absorption elements.

\begin{quote}
we need to do some market research on the Q12: 1. what's the feedback report look like? (google images show one overall ``satsifaction'' score and/or one overall ``engagement'' score), 2. how much does it cost, 3. what are the 200 pulse items Gallup refers to? (6/7/21)
\end{quote}

Our conceptualization of work engagement is a mental state wherein employees\ldots{}

\begin{itemize}
\tightlist
\item
  \ldots feel energized (\textbf{Vigor})
\item
  \ldots are enthusiastic about the content of their work and the things they do (\textbf{Dedication})
\item
  \ldots are so immersed in their work activities that time seems compressed (\textbf{Absorption})
\end{itemize}

\hypertarget{methods}{%
\section{Methods}\label{methods}}

We report how we determined our sample size, all data exclusions (if any), all manipulations, and all measures in the study.

\hypertarget{participants}{%
\subsection{Participants}\label{participants}}

\hypertarget{material}{%
\subsection{Material}\label{material}}

\hypertarget{procedure}{%
\subsection{Procedure}\label{procedure}}

Looking into the specification of polychoric covariances (Jöreskog, 1994). This seems to be not very commonly leveraged (only package that seems to estimate these is \texttt{semPlot})

\hypertarget{data-analysis}{%
\subsection{Data analysis}\label{data-analysis}}

We used R {[}Version 4.1.0; R Core Team (2021){]} and the R-packages \emph{dplyr} {[}Version 1.0.6; Wickham, François, Henry, and Müller (2021){]}, \emph{kableExtra} {[}Version 1.3.4; Zhu (2021){]}, \emph{lavaan} {[}Version 0.6.8; Rosseel (2012){]}, \emph{papaja} {[}Version 0.1.0.9997; Aust and Barth (2020){]}, \emph{sem} {[}Version 3.1.11; Fox, Nie, and Byrnes (2020); Epskamp (2019){]}, and \emph{semPlot} {[}Version 1.1.2; Epskamp (2019){]} for all our analyses.

\hypertarget{results}{%
\section{Results}\label{results}}

\hypertarget{discussion}{%
\section{Discussion}\label{discussion}}

\newpage

\hypertarget{references}{%
\section{References}\label{references}}

\begingroup
\setlength{\parindent}{-0.5in}
\setlength{\leftskip}{0.5in}

\hypertarget{refs}{}
\begin{CSLReferences}{1}{0}
\leavevmode\hypertarget{ref-R-papaja}{}%
Aust, F., \& Barth, M. (2020). \emph{{papaja}: {Create} {APA} manuscripts with {R Markdown}}. Retrieved from \url{https://github.com/crsh/papaja}

\leavevmode\hypertarget{ref-coffman_hard_1999}{}%
Coffman, C., \& Harter, J. (1999). A hard look at soft numbers. \emph{Position Paper, Gallup Organization}.

\leavevmode\hypertarget{ref-cole2012job}{}%
Cole, M. S., Walter, F., Bedeian, A. G., \& O'Boyle, E. H. (2012). Job burnout and employee engagement: A meta-analytic examination of construct proliferation. \emph{Journal of Management}, \emph{38}(5), 1550--1581.

\leavevmode\hypertarget{ref-elloy_examination_1991}{}%
Elloy, D. F., Everett, J. E., \& Flynn, W. R. (1991). An examination of the correlates of job involvement. \emph{Group \& Organization Studies}, \emph{16}(2), 160--177. \url{https://doi.org/10.1177/105960119101600204}

\leavevmode\hypertarget{ref-R-semPlot}{}%
Epskamp, S. (2019). \emph{semPlot: Path diagrams and visual analysis of various SEM packages' output}. Retrieved from \url{https://CRAN.R-project.org/package=semPlot}

\leavevmode\hypertarget{ref-ferris_added_1984}{}%
Ferris, R., \& Hellier, P. (1984). Added value productivity schemes and employee participation. \emph{Asia Pacific Journal of Human Resources}, \emph{22}(4), 35--44. \url{https://doi.org/10.1177/103841118402200406}

\leavevmode\hypertarget{ref-R-sem}{}%
Fox, J., Nie, Z., \& Byrnes, J. (2020). \emph{Sem: Structural equation models}. Retrieved from \url{https://CRAN.R-project.org/package=sem}

\leavevmode\hypertarget{ref-harter_business-unit-level_2002}{}%
Harter, J. K., Schmidt, F., \& Hayes, T. (2002). Business-unit-level relationship between employee satisfaction, employee engagement, and business outcomes: A meta-analysis. \emph{The Journal of Applied Psychology}. \url{https://doi.org/10.1037/0021-9010.87.2.268}

\leavevmode\hypertarget{ref-joreskog1994estimation}{}%
Jöreskog, K. G. (1994). On the estimation of polychoric correlations and their asymptotic covariance matrix. \emph{Psychometrika}, \emph{59}(3), 381--389.

\leavevmode\hypertarget{ref-kahn_psychological_1990}{}%
Kahn, W. A. (1990). Psychological conditions of personal engagement and disengagement at work. \emph{Academy of Management Journal}, \emph{33}(4), 692--724.

\leavevmode\hypertarget{ref-leiter_areas_2004}{}%
Leiter, M., \& Maslach, C. (2004). Areas of worklife: A structured approach to organizational predictors of job burnout. In \emph{Research in occupational stress and well-being} (Vol. 3, pp. 91--134). \url{https://doi.org/10.1016/S1479-3555(03)03003-8}

\leavevmode\hypertarget{ref-leone_relation_1995}{}%
Leone, D. R. (1995). \emph{The relation of work climate, higher order need satisfaction, need salience, and causality orientations to work engagement, psychological adjustment, and job satisfaction} (PhD thesis). ProQuest Information \& Learning.

\leavevmode\hypertarget{ref-maslach1997causes}{}%
Maslach, C., \& Leiter, M. (1997). What causes burnout. \emph{Maslach C, Leiter MP. The Truth About Burnout: How Organizations Cause Personal Stress and What to Do about It. San Francisco, CA: Josey-Bass Publishers}, 38--60.

\leavevmode\hypertarget{ref-maslach_early_2008}{}%
Maslach, Christina, \& Leiter, M. P. (2008). Early predictors of job burnout and engagement. \emph{Journal of Applied Psychology}, \emph{93}(3), 498--512.

\leavevmode\hypertarget{ref-meyer_three-component_1991}{}%
Meyer, J. P., \& Allen, N. J. (1991). A three-component conceptualization of organizational commitment. \emph{Human Resource Management Review}, \emph{1}(1), 61--89.

\leavevmode\hypertarget{ref-R-base}{}%
R Core Team. (2021). \emph{R: A language and environment for statistical computing}. Vienna, Austria: R Foundation for Statistical Computing. Retrieved from \url{https://www.R-project.org/}

\leavevmode\hypertarget{ref-R-lavaan}{}%
Rosseel, Y. (2012). {lavaan}: An {R} package for structural equation modeling. \emph{Journal of Statistical Software}, \emph{48}(2), 1--36. Retrieved from \url{https://www.jstatsoft.org/v48/i02/}

\leavevmode\hypertarget{ref-schaufeli_uwesutrecht_2003}{}%
Schaufeli, W. B., \& Bakker, A. B. (2003). {UWES}--utrecht work engagement scale: Test manual. \emph{Unpublished Manuscript: Department of Psychology, Utrecht University}, \emph{8}.

\leavevmode\hypertarget{ref-schaufeli_measurement_2002}{}%
Schaufeli, Wilmar B., Salanova, M., González-Romá, V., \& Bakker, A. B. (2002). The measurement of engagement and burnout: A two sample confirmatory factor analytic approach. \emph{Journal of Happiness Studies}, \emph{3}(1), 71--92.

\leavevmode\hypertarget{ref-schaufeli_conceptualization_2010}{}%
Schaufeli, W., \& Bakker, A. (2010). The conceptualization and measurement of work engagement. In W. Schaufeli, A. Bakker, \& M. Leiter (Eds.), \emph{Work engagement: A handbook of essential theory and research} (pp. 10--24). New York: Psychology Press.

\leavevmode\hypertarget{ref-staw_employee_1994}{}%
Staw, B. M., Sutton, R. I., \& Pelled, L. H. (1994). Employee positive emotion and favorable outcomes at the workplace. \emph{Organization Science}, \emph{5}(1), 51--71.

\leavevmode\hypertarget{ref-taris2017burnout}{}%
Taris, T. W., Ybema, J. F., \& Beek, I. van. (2017). Burnout and engagement: Identical twins or just close relatives? \emph{Burnout Research}, \emph{5}, 3--11.

\leavevmode\hypertarget{ref-timms2012burnt}{}%
Timms, C., Brough, P., \& Graham, D. (2012). Burnt-out but engaged: The co-existence of psychological burnout and engagement. \emph{Journal of Educational Administration}, \emph{50}(3), 327--345.

\leavevmode\hypertarget{ref-R-dplyr}{}%
Wickham, H., François, R., Henry, L., \& Müller, K. (2021). \emph{Dplyr: A grammar of data manipulation}. Retrieved from \url{https://CRAN.R-project.org/package=dplyr}

\leavevmode\hypertarget{ref-R-kableExtra}{}%
Zhu, H. (2021). \emph{kableExtra: Construct complex table with 'kable' and pipe syntax}. Retrieved from \url{https://CRAN.R-project.org/package=kableExtra}

\end{CSLReferences}

\endgroup


\end{document}
